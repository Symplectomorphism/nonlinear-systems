\section{Linear Velocity of a Point Attached to a Moving Frame}

\begin{frame}[standout, plain, noframenumbering]
    Linear Velocity of a Point Attached to a Moving Frame

    % \medskip

    % \footnotesize
    % Sam Greydanus \quad Misko Dzamba \quad Jason Yosinski
\end{frame}

\begingroup
\small

\begin{frame}
    \frametitle{Linear Velocity of a Point Attached to a Moving Frame}

    \begin{itemize}
        \item Suppose the point $p$ is rigidly attached to $\Sigma_1$ and that 
        $\Sigma_1$ is rotating relative to the frame $\Sigma_0$.
        \[ p^0 = R_1^0(t)p^1. \]
        \item Differentiating this equation, we get 
        \[ \dot{p}^0 = \dot{R}_1^0 p^1 + \cancelto{0}{R_1^0 \dot{p}^1} =
        S\left( \omega^0 \right) R_1^0p^1 = S\left( \omega^0 \right) p^0 =
        \omega^0 \times p. \]
    \end{itemize}
\end{frame}


\begin{frame}
    \frametitle{Linear Velocity of a Point Attached to a Moving Frame}

    \begin{itemize}
        \item Now suppose that the motion of $\Sigma_1$ w.r.t. $\Sigma_0$ is
        more general and is given by \[ H_1^0(t) = \bmat{
            R(t) & o(t) \\ 0 & 1
        }. \]
        \item Hence we have $\qquad p^0 = Rp^1 + o$.
        \item Differentiating, we obtain,
        \[ \dot{p}^0 = \dot{R}p^1 + \dot{o} = S(\omega)Rp^1 + \dot{o} = \omega
        \times r + v, \] where $r = Rp^1$ is the vector from $o_1$ to $p$,
        expressed in $\Sigma_0$, and $v$ is the rate at which the origin $o_1$ 
        is moving.
        \item If the point $p$ is moving relative to $\Sigma_1$, then we must
        add to the term $v$ the term $R(t)\dot{p}^1$, which is the rate of
        change of the coordinates $p^1$, expressed in $\Sigma_0$.
    \end{itemize}
\end{frame}

\endgroup