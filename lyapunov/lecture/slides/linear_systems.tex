\section{Stability of Linear Systems}

\begin{frame}[standout, plain, noframenumbering]
    Stability of Linear Systems

    % \medskip

    % \footnotesize
    % Sam Greydanus \quad Misko Dzamba \quad Jason Yosinski
\end{frame}

\begingroup
\small



\begin{frame}
    \frametitle{Autonomous Linear Systems}

    We restrict our attention to linear \textit{autonomous} systems of the form 
    \begin{equation}
        \dot{x}(t) = A x(t).
        \label{eq:lin_system}
    \end{equation}

    \begin{theorem}
        The equilibrium $0$ of~\eqref{eq:lin_system} is (globally) exponentially
        stable if and only if all eigenvalues of $A$ have negative real parts.
        The equilibrium is stable if and only if all eigenvalues of $A$ have
        nonpositive real parts, and in addition, every eigenvalues of $A$ having
        a zero real part is a simple zero of the minimal polynomial of $A$.
    \end{theorem}
\end{frame}

\begin{frame}
    \frametitle{Lyapunov Function}

    Given the system~\eqref{eq:lin_system}, we choose a Lyapunov function
    candidate: \[ V(x) = x^\top P x \implies \dot{V} = \dot{x}^\top P x + x^\top
    P \dot{x} = - x^\top Q x, \] where $P = P^\top$ and 
    \begin{equation}
        A^\top P + PA = -Q.
        \label{eq:lyap_matrix}
    \end{equation}
    Equation~\eqref{eq:lyap_matrix} is commonly known as the \textbf{Lyapunov
    Matrix Equation}.
    \begin{rem}[Stability]
        If a pair of matrices $(P, Q)$ satisfying~\eqref{eq:lyap_matrix} can be
        found such that both $P$ and $Q$ are positive definite, then both $V$
        and $-\dot{V}$ are positive definite functions and $V$ is radially
        unbounded. Hence, the equilibrium $0$ is globally exponentially stable.

        If a pair $(P, Q)$ can be found s.t. $Q > 0$ and $P$ has at least one
        nonpositive eigenvalue, then $-\dot{V} > 0$ and $V$ assumes nonpositive
        values arbitrarily close to the origin. Hence $0$ is unstable.
    \end{rem}
\end{frame}


\begin{frame}
    \frametitle{Lyapunov Matrix Equation}

    \begin{lemma}
        Let $\{\lambda_i\}_1^n$ denote the eigenvalues of $A$. Then
        equation~\eqref{eq:lyap_matrix} has a unique solution for $P$
        corresponding to each $Q \in \mathbb{R}^{n \times n}$ iff \[ \lambda_i +
        \lambda_j \neq 0, \;\; \forall i, j. \]
    \end{lemma}

    \begin{corollary}
        If for some $Q \in \mathbb{R}^{n \times n}$ does not have a unique
        solution for $P$, then the origin is not an asymptotically stable
        equilibrium.
    \end{corollary}
    \begin{proof}
        If all eigenvalues of $A$ has negative real parts, then the equation 
        above is satisfied. \hfill \qed
    \end{proof}
\end{frame}


\begin{frame}
    \frametitle{Main Result}

    \begin{theorem}
        Given a matrix $A \in \mathbb{R}^{n \times n}$, the following are
        equivalent:
        \begin{itemize}
            \item $A$ is a Hurwitz matrix (all its e.vals have negative real parts).
            \item There exists \textsc{some} $Q \in \mathbb{S}_{++}^n$ such that
            equation~\eqref{eq:lyap_matrix} has a corresponding unique solution
            for $P \in \mathbb{S}_{++}^n$.
            \item For \textsc{every} $Q \in \mathbb{S}_{++}^n$,
            equation~\eqref{eq:lyap_matrix} has a unique solution for $P \in
            \mathbb{S}_{++}^n$.
        \end{itemize}
    \end{theorem}
    \begin{proof}
        ``$(3) \implies (2)$'' Obvious. \\
        ``$(2) \implies (1)$'' Suppose $(2)$ is true for some particular matrix
        $Q$. Consider the candidate $V(x) = x^\top P x$. Then $\dot{V}(x) =
        -x^\top Q x$, and one can conclude that $0$ is asymptotically stable. 
        Hence $A$ is Hurwitz. \\
        ``$(1) \implies (3)$'' Omitted (see Section 5.4, Theorem (42) in
        Vidyasagar, ``\textit{Nonlinear Systems Analysis}'', 1993.)
    \end{proof}
\end{frame}


\endgroup