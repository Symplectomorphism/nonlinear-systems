\section{Introduction}

\begin{frame}[standout, plain, noframenumbering]
    Introduction

    % \medskip

    % \footnotesize
    % Sam Greydanus \quad Misko Dzamba \quad Jason Yosinski
\end{frame}

\begin{frame}
    \frametitle{Lyapunov Stability -- Introduction}

    

    \begin{itemize}
        \item Introduced by Alexandr Mikhailovich Lyapunov.
        \item \textit{The general problem of the stability of motion}, 1892.
        \item Doctoral thesis in Kharkov Mathematical Society.
        \item The most general theory for analyzing stability of (at least)
        ordinary differential equations.
    \end{itemize}

\end{frame}


\begin{frame}
    \frametitle{Lyapunov Stability -- Introduction}

    \begin{itemize}
        \item Different notions of stability: input-output stability, periodic
        orbit stability, etc. 
        \item Stability of equilibrium points usually characterized in the sense 
        of Lyapunov.
        \begin{itemize}
            \item An equilibrium point is \textsc{stable} if all solutions
            starting at nearby points stay nearby.
            \item It is \textsc{asymptotically stable} if all solutions starting
            at nearby points not only stay nearby, but also tend to the
            equilibrium point as time approaches infinity.
        \end{itemize}
        \item For a linear system $\dot{x} = Ax$, the stability of $x=0$ can be
        completely characterized by the eigenvalues of $A$.
        \item Stability of a nonlinear system sometimes can be characterized by
        the same method (through linearization).
        \item Lyapunov stability theorems give sufficient conditions for
        stability.
    \end{itemize}
\end{frame}