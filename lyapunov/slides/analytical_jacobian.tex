\section{The Analytical Jacobian}

\begin{frame}[standout, plain, noframenumbering]
    The Analytical Jacobian

    % \medskip

    % \footnotesize
    % Sam Greydanus \quad Misko Dzamba \quad Jason Yosinski
\end{frame}

\begingroup
\small



\begin{frame}
    \frametitle{The Analytical Jacobian}

    \begin{itemize}
        \item The Jacobian matrix derived above is sometimes called the
        \textbf{geometric Jacobian} to distinguish from the \textbf{analytical
        Jacobian}, denoted $J_a(q)$. Let \[ X = \bmat{d(q) \\ \alpha(q)} \]
        denote the end effector pose
        \begin{itemize}
            \footnotesize{
            \item $d(q)$ is the usual vector from the origin of the base frame to the origin of the end-effector frame,
            \item $\alpha(q)$ denotes a minimal representation for the orientation of the end effector frame relative to the base frame.
            }
        \end{itemize} 
        \item Let $\alpha = (\phi, \theta, \psi)$ be a vector of Euler angles.
        Then, we seek an expression of the form \[ \dot{X} = \bmat{\dot{d} \\
        \dot{\alpha}} = J_a(q)\dot{q} \] to define the analytical Jacobian.
    \end{itemize}
\end{frame}


\begin{frame}
    \frametitle{The Analytical Jacobian}

    \begin{itemize}
        \item It can be shown that, if $R = R_{z,\phi}R_{y,\theta}R_{z,\psi}$ is 
        the Euler angle transformation, then \[ \dot{R} = S(\omega)R \] in which 
        $\omega$ defining the angular velocity is given by
        \[
        \omega = \bmat{c_\psi s_\theta \dot{\phi} - s_\psi \dot{\theta} \\ 
                    s_\psi s_\theta \dot{\phi} + c_\psi \dot{\theta} \\ 
                    \dot{\psi} + c_\theta \dot{\phi}} = 
        \bmat{
            c_\psi s_\theta & -s_\psi & 0 \\ s_\psi s_\theta & c_\psi & 0 \\ 
            c_\theta & 0 & 1
        }\bmat{\dot{\phi} \\ \dot{\theta} \\ \dot{\psi}} = B(\alpha)\dot{\alpha}.
        \]
        \item The components of $\omega$ are called \textbf{nutation},
        \textbf{spin}, and \textbf{precession}.
        \item Combining the above relationships with the previous definition
        yields
        \[
        J(q)\dot{q} = \bmat{v \\ \omega} = \bmat{\dot{d} \\ B(\alpha)\dot{\alpha}}
        = \bmat{I & 0 \\ 0 & B(\alpha)}\bmat{\dot{d} \\ \dot{\alpha}} = \bmat{I & 0 \\ 0 & B(\alpha)}J_a(q)\dot{q}.
        \]
    \end{itemize}
\end{frame}


\begin{frame}
    \frametitle{The Analytical Jacobian}

    \begin{itemize}
        \item Thus, the analytical Jacobian, $J_a(q)$, may be computed from the 
        geometric Jacobian as
        \[ J_a(q) = \bmat{I & 0 \\ 0 & B^{-1}(\alpha)}J(q), \]
        provided that $\det B(\alpha) \neq 0$.
        \item In the next section, we discuss the notion of Jacobian
        singularities, which are configurations where the Jacobian loses rank.
        \item Singularities of the matrix $B(\alpha)$ are called
        \textbf{representation singularities}.
        \item It can be easily shown that $B(\alpha)$ is invertible provided
        that $s_\theta \neq 0$.
        \item The singularities of the analytical Jacobian include the
        singularities of the geometric Jacobian, $J$, together with the 
        representational singularities.
    \end{itemize}
\end{frame}


\endgroup