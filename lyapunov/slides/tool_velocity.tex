\section{The Tool Velocity}

\begin{frame}[standout, plain, noframenumbering]
    The Tool Velocity

    % \medskip

    % \footnotesize
    % Sam Greydanus \quad Misko Dzamba \quad Jason Yosinski
\end{frame}

\begingroup
\small

\begin{frame}
    \frametitle{The Tool Velocity}

    \begin{itemize}
        \item Many tasks require that a tool be attached to the end effector.
        \item It is necessary to relate the velocity of the tool frame to the 
        velocity of the end-effector frame.
        \item The fixed spatial relationship between the end effector and the 
        tool frame is given by the constant homogeneous transformation matrix 
        \[
        T_{\textrm{tool}}^n = \bmat{R & d \\ 0 & 1}.
        \]
        \item We assume that the end effector velocity is given and expressed in
        the end effector frame $\Sigma_n$, i.e., we are given $\xi_n^n$.
        \item We will derive the velocity of the tool expressed in coordinates
        relative to the tool frame, that is, we will derive
        $\xi_{\textrm{tool}}^{\textrm{tool}}$.
    \end{itemize}
\end{frame}



\begin{frame}
    \frametitle{The Tool Velocity}

    \begin{itemize}
        \item Since the two frames are rigidly attached, the angular velocity of
        the tool frame is the same as the angular velocity of the end-effector
        frame. Indeed, we have $R_{\textrm{tool}}^0 = R_n^0R$, with $R$ constant
        \[
            \dot{R}_{\textrm{tool}}^0 = \dot{R}_n^0R \Longrightarrow 
            S\left( \omega_{\textrm{tool}}^0 \right)R_{\textrm{tool}}^0 = S\left( \omega_n^0 \right)R_n^0R
            \Longrightarrow \omega_{\textrm{tool}} = \omega_n.
        \]
        \item To obtain the tool angular velocity relative to the tool frame, we
        apply a rotational transformation \[
        \omega_{\textrm{tool}}^{\textrm{tool}} = \omega_n^{\textrm{tool}} =
        R^\top \omega_n^n. \]
        \item If the end-effector frame is moving with body velocity $\xi =
        (v_n, \omega_n)$, then the linear velocity of the origin of the tool
        frame, which is rigidly attached to the end-effector frame, is given by
        \[ v_{\textrm{tool}} = v_n + \omega_n \times r, \] where $r$ is the
        vector from the origin of $\Sigma_n$ to the origin of
        $\Sigma_{\textrm{tool}}$.
    \end{itemize}
\end{frame}


\begin{frame}
    \frametitle{The Tool Velocity}

    \begin{itemize}
        \item We can express $r$ in coordinates relative to
        $\Sigma_{\textrm{tool}}$ as $r^{\textrm{tool}} = R^\top d$.
        \item Thus, we can write 
        \begin{align*}
            \omega_n^{\textrm{tool}} \times r^{\textrm{tool}} &= 
            R^\top \omega_n^n \times \left( R^\top d \right) = -R^\top d \times 
            R^\top \omega_n^n = -S\left( R^\top d \right)R^\top \omega_n^n \\
            &= R^\top S(d) RR^\top \omega_n^n = -R^\top S(d) \omega_n^n.
        \end{align*}
        \item We express the free vector $v_n^n$ in coordinates relative to
        $\Sigma_{\textrm{tool}}$ \[v_n^{\textrm{tool}} = R^\top v_n^n. \]
        \item Combining, we obtain
        \begin{align*}
            v_{\textrm{tool}}^{\textrm{tool}} &= R^\top v_n^n - R^\top S(d) \omega_n^n, \\
            \omega_{\textrm{tool}}^{\textrm{tool}} &= R^\top \omega_n^n.
        \end{align*}
        which can be expressed as the matrix equation
        \[
        \xi_{\textrm{tool}}^{\textrm{tool}} = \bmat{R^\top & -R^\top S(d) \\ 0_{3 \times 3} & R^\top}\xi_n^n.
        \]
    \end{itemize}
\end{frame}


\begin{frame}
    \frametitle{The Tool Velocity}

    \begin{itemize}
        \item In many cases, it is useful to solve the inverse problem: compute
        the required end effector velocity to produce a desired tool velocity.
        Since 
        \[
        \bmat{R & S(d)R \\ 0 & R} = \bmat{R^\top & -R^\top S(d) \\ 0 & R^\top}^{-1}
        \]
        we can solve the equation above for $\xi_n^n$, obtaining
        \[ \omega_n^n = \bmat{R & S(d)R \\ 0 & R}\xi_{\textrm{tool}}^{\textrm{tool}}. \]
        \item This gives the general expression for transforming velocities
        between two rigidly attached moving frames
        \[
        \xi_A^A = \bmat{R_B^A & S\left( d_B^A \right)R_B^A \\ 0 & R_B^A}\xi_B^B.
        \]
    \end{itemize}
\end{frame}


\endgroup