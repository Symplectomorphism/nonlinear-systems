\section{Lyapunov Stability Analysis on Euclidean Spaces}

\begin{frame}[standout, plain, noframenumbering]
    Lyapunov Stability Analysis on Euclidean Spaces

    % \medskip

    % \footnotesize
    % Sam Greydanus \quad Misko Dzamba \quad Jason Yosinski
\end{frame}

\begingroup
\small

\begin{frame}
    \frametitle{Autonomous Systems}

    Consider the autonomous system 
    \begin{equation}
        \dot{x} = f(x)
        \label{eq:autonomous}
    \end{equation}
    where $f: D \subseteq \mathbb{R}^n \rightarrow \mathbb{R}^n$ is a locally
    Lipschitz map, with an equilibrium point at $x = 0$.

    \begin{definition}
        The equilibrium point $x=0$ of the system~\eqref{eq:autonomous} is
        \begin{itemize}
            \item \textit{stable} if, $\forall \varepsilon > 0$, $\exists \delta
            = \delta(\varepsilon) > 0$ such that 
            \[
                \norm{x(0)}{} < \delta \; \Rightarrow \; \norm{x(t)}{} < 
                \epsilon, \; \; \forall t \geq 0.
            \]
            \item \textit{unstable} if it is not stable.
            \item \textit{asymptotically stable} if it is stable and $\delta$
            can be chosen s.t.
            \[ \norm{x(0)}{} < \delta \; \Rightarrow \; \lim_{t \to \infty} x(t)
            = 0. \]
        \end{itemize}
    \end{definition}
\end{frame}


\begin{frame}
    \frametitle{Example -- Pendulum}

    The pendulum equation
    \begin{align*}
        \dot{x}_1 &= x_2 \\
        \dot{x}_2 &= -a \sin{x_1} - b x_2
    \end{align*}
    has two equilibrium points at ($x_1 = 0$, $x_2 = 0$) and ($x_1 = \pi$, $x_2
    = 0$).
    %
    \begin{itemize}
        \item If $b=0$, trajectories in the nbhd. of the first equilibrium are 
        closed orbits.
        \item By starting sufficiently close to the eq. point, trajectories are 
        guaranteed to stay within any specified ball.
        \item The point is not asymptotically stable since trajectories don't
        tend to the eq. point.
        \item If $b > 0$, the origin becomes asymptotically stable.
        \item The second eq. point is a saddle point: the $\varepsilon-\delta$
        requirement cannot be satisfied (for every $\varepsilon > 0$ there
        exists a trajectory that will leave the ball $B_\varepsilon$ even if
        $x(0)$ is arbitrarily close to $(\pi, 0)$).
    \end{itemize}
\end{frame}


\begin{frame}
    \frametitle{Lyapunov Stability Theorem}

    \begin{theorem}
        Let $x=0 \in D$ be an equilibrium point for~\eqref{eq:autonomous}. Let
        $V: D \rightarrow \mathbb{R}$ be a continuously differentiable function
        such that
        \begin{align*}
            V(0) = 0 \; &\text{ and } \; V(x) > 0 \text{ in } D - \{0\}, \\
            &\dot{V}(x) \leq 0 \text{ in } D.
        \end{align*}
        Then, $x=0$ is stable. Moreover, if
        \[ \dot{V}(x) < 0 \text{ in } D - \{0\} \] then $x=0$ is asymptotically
        stable.
    \end{theorem}
\end{frame}


\begin{frame}
    \frametitle{Lyapunov Stability Theorem}

    \begin{proof}[Proof of stability]
        Given $\varepsilon > 0$, choose $0 < r \leq \varepsilon$ such that $B_r
        \subseteq D$. Let $\alpha = \min_{\norm{x}{}=r} V(x)$. Then, $\alpha >
        0$. Take $0 < \beta < \alpha$ and consider $\mc{M}_\beta = V^{-1}((0,
        \beta])$.

        \underline{Claim}: $\mc{M}_\beta \subseteq \accentset{\circ}{B}_r$.
        Argue ad absurdum. Suppose $\mc{M}_\beta \cap \accentset{\circ}{B}_r
        \neq \mc{M}_\beta$. Then $\exists p \in \mc{M}_\beta \cap \partial B_r$.
        Note, $V(p) \geq \alpha > \beta$, but $V(\mc{M}_\beta) \subseteq
        [0,\beta]$.

        The set $\mc{M}_\beta$ is invariant since \[ \dot{V}(x(t)) \leq 0 \;
        \Rightarrow \; V(x(t)) \leq V(x(0)) \leq \beta, \; \forall t \geq 0. \]

        Because $\mc{M}_\beta$ is compact (closed and bounded), we conclude that
        the ODE~\eqref{eq:autonomous} has a unique solution $\forall t \geq 0$
        whenever $x(0) \in \mc{M}_\beta$. Since $V$ is continuous and $V(0) =
        0$, $\exists \delta > 0$ such that \[ \norm{x}{} \leq \delta \;
        \Rightarrow \; V(x) < \beta. \]
    \end{proof}
\end{frame}

\begin{frame}
    \frametitle{Lyapunov Stability Theorem}

    \begin{proof}[Proof of stability (cont'd)]
        Then, \[ B_\delta \subseteq \mc{M}_\beta \subseteq B_r \] and 
        \[ x(0) \in B_\delta \; \Rightarrow \; x(0) \in \mc{M}_\beta \;
        \Rightarrow \; x(t) \in \mc{M}_\beta \; \Rightarrow \; x(t) \in B_r, \]
        proving stability.
    \end{proof}
\end{frame}

\begin{frame}
    \frametitle{Lyapunov Stability Theorem}

    \begin{proof}[Proof of asymptotic stability]
        Now assume $\dot{V}(x) < 0$ in $D - \{0\}$. We want to show that $x(t)
        \xrightarrow{t \to \infty} 0$; i.e., $\forall a > 0$, $\exists T > 0$, 
        s.t. $\norm{x(t)}{} < a, \forall t > T$.

        We know that $\forall a > 0$, we can choose $b > 0$ s.t. $\mc{M}_b
        \subseteq B_a$. Therefore, it is sufficient to show that $V(x(t))
        \xrightarrow{t \to \infty} 0$. Since $V$ is monotonically decreasing and
        bounded from below by zero, \[ V(x(t)) \xrightarrow{t \to \infty} c \geq
        0. \]

        \underline{Claim}: $c = 0$. Argue ad absurdum. Suppose $c > 0$. By
        continuity of $V$, $\exists d > 0$ s.t. $B_d \subseteq \mc{M}_c$. The
        limit $V(x(t)) \rightarrow c > 0$ implies that $x(t) \notin B_d, \forall
        t \geq 0$. Define $\max_{d \leq \norm{x}{} \leq r} \dot{V}(x) =: -\gamma
        < 0$. It follows that 
        \[
        V(x(t)) = V(x(0)) + \int_0^t \dot{V}(x(\tau)) \dd \tau \leq V(x(0)) - \gamma t.
        \]
        The RHS will eventually become negative: contradiction ($c > 0$).
    \end{proof}
\end{frame}

\endgroup