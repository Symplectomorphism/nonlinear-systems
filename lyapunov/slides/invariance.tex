\section{The Invariance Principle}

\begin{frame}[standout, plain, noframenumbering]
    The Invariance Principle

    % \medskip

    % \footnotesize
    % Sam Greydanus \quad Misko Dzamba \quad Jason Yosinski
\end{frame}

\begingroup
\small


\begin{frame}
    \frametitle{Intuition: Damped Pendulum}

    \begin{columns}
        \begin{column}{0.5\textwidth}
            \begin{align*}
                \dot{x}_1 &= x_2, \\
                \dot{x}_2 &= -a \sin{x_1} - bx_2^2.
            \end{align*}
        \end{column}
        \begin{column}{0.5\textwidth}
            \underline{Lyapunov function candidate}\\[0.75ex]
            $ V(x) = a(1 - \cos{x_1}) + \frac{1}{2}x_2^2. $
        \end{column}
    \end{columns}
    \[ \dot{V}(x) = -bx_2^2 \leq 0. \]

    \begin{itemize}
        \item $\dot{V}(x) < 0$ if and only if $x_2 \neq 0$.
        \item For the system to maintain $\dot{V}(x) = 0$, it has to stay on
        $x_2=0$.
        \item Unless $x_1=0$, this is impossible:
        \[ x_2(t) \equiv 0 \; \Rightarrow \; \dot{x}_2 \equiv 0 \; \Rightarrow
        \; \sin{x_1(t)} \equiv 0. \]
        \item Hence, on the segment $-\pi < x_1 < \pi$ of the $x_2=0$ line, the
        system can maintain $\dot{V}(x) = 0$ only at the origin $x=0$.
        \item Therefore, $V(x(t))$ must decrease towards $0$ and, consequently, 
        \[ x(t) \xrightarrow{t \to \infty} 0. \]
    \end{itemize}
\end{frame}


\begin{frame}
    \frametitle{Limit and Invariant Sets}

    \begin{definition}[Limit points and limit sets]
        A point $p$ is said to be a \textit{positive limit point} of $x(t)$ if 
        there is a sequence $\{t_n\}$, with $t_n \to \infty$ as $n \to \infty$, 
        such that $x(t_n) \to p$ as $n \to \infty$.

        The set of all positive limit points of $x(t)$ is called the
        \textit{positive limit set} of $x(t)$.
    \end{definition}

    \begin{definition}[Positively Invariant Set]
        A set $M$ is said to be an \textit{invariant set}
        w.r.t.~\eqref{eq:autonomous} if\\[0.5ex] \hspace{29mm} $ x(0) \in M \;
        \Rightarrow \; x(t) \in M, \; \forall t \in \mathbb{R}. $\\[0.5ex] That
        is, if a solution belongs to $M$
        at some time instant, then it belongs to $M$ for all future and past
        time.

        A set $M$ is said to be a \textit{positively invariant set} if 
        \[ x(0) \in M \; \Rightarrow \; x(t) \in M, \; \forall t \geq 0. \]
    \end{definition}
\end{frame}


\begin{frame}
    \frametitle{Distance to an (Invariant) Set }

    \begin{definition}[Distance and Convergence to a Set]
        We say that $x(t)$ approaches a set $M$ as $t \to \infty$, if for each 
        $\varepsilon > 0$, $\exists T > 0$ such that\\[0.5ex] 
        \hspace{22mm} $ \inf_{x \in M} \norm{p-x}{} =: \text{dist}(x(t),  M) <
        \varepsilon, \; \forall t > T. $\\[0.5ex]
    \end{definition}
    %
    \vspace{-3mm}
    \begin{itemize}
        \item An asymptotically stable equilibrium point is the positive limit
        set of every solution starting sufficiently near the equilibrium point.
        \item A stable limit cycle is the positive limit set of every solution 
        starting sufficiently near the limit cycle.
        \item The solution approaches the limit cycle as $t \to \infty$. Notice:
        the solution does not approach any specific point on the limit cycle.
        \item The statement $x(t)$ approaches $M$ as $t \to \infty$ does not
        imply that $\lim_{t \to \infty} x(t)$ exists.
        \item The set $\mc{M}_c = \{x \in \mathbb{R}^n: V(x) \leq c\}$ with
        $\dot{V}(x) \leq 0$ for all $x \in \mc{M}_c$ is a positively invariant
        set.
    \end{itemize}
\end{frame}


\begin{frame}
    \frametitle{Limit Sets and Krasovskii-LaSalle Theorem}

    \begin{lemma}
        If a solution $x(t)$ is bounded and belongs to $D$ for $t \geq 0$, then
        its positive limit set $L^+$ is a nonempty, compact, invariant set.
        Moreover, $x(t)$ approaches $L^+$ as $t \to \infty$.
    \end{lemma}

    \begin{theorem}[Krasovskii-LaSalle Theorem]
        Let $\Omega \subseteq D$ be a compact set that is positively invariant
        w.r.t.~\eqref{eq:autonomous}. Let $V: D \rightarrow \mathbb{R}$ be a
        continuously differentiable function such that $\dot{V}(x) \leq 0$ in
        $\Omega$. Let $E$ be the set of all points in $\Omega$ where $\dot{V}(x)
        = 0$. Let $M$ be the largest invariant set in $E$. Then every solution
        starting in $\Omega$ approaches $M$ as $t \to \infty$.
    \end{theorem}
\end{frame}

\begin{frame}
    \frametitle{Krasovskii-LaSalle Theorem}

    \begin{proof}
        Let $x(t)$ be a solution of~\eqref{eq:autonomous} starting in $\Omega$.
        Since $\dot{V}(x) \leq 0$ in $\Omega$, $V(x(t))$ is a decreasing
        function of $t$. Since $V(x)$ is continuous on the compact set $\Omega$,
        it is bounded from below on $\Omega$. Therefore, $V(x(t))$ has a limit
        $a$ as $t \to \infty$. Note that the positive limit set $L^+$ is in
        $\Omega$ because $\Omega$ is a closed set. For any $p \in L^+$, there is
        a sequence $t_n$ with $t_n \to \infty$ and $x(t_n) \to p$ as $n \to
        \infty$. By the continuity of $V(x)$, $V(p) = \lim_{n \to \infty}
        V(x(t_n)) = a$. Hence, $V(x) = a$ on $L^+$. Since $L^+$ is an invariant
        set, $\dot{V}(x) = 0$ on $L^+$. Thus, \[ L^+ \subseteq M \subseteq E
        \subseteq \Omega \] Since $x(t)$ is bounded, $x(t)$ approaches $L^+$ as 
        $t \to \infty$. Hence, $x(t)$ approaches $M$ as $t \to \infty$. \hfill 
        \qed
    \end{proof}
\end{frame}

\begin{frame}
    \frametitle{Krasovskii-LaSalle Theorem}

    \begin{itemize}
        \item Notice that, this theorem does not require the function $V(x)$ to 
        be positive definite.
        \item The set $\Omega$ does not have to be tied in with the construction
        of the function $V(x)$.
        \item However, in many applications, the construction of $V(x)$ will
        itself guarantee the existence of a set $\Omega$. In particular, if
        $\mc{M}_c = \{x \in \mathbb{R}^n: V(x) \leq c\}$ is bounded and
        $\dot{V}(x) \leq 0$ in $\mc{M}_c$, then we can take $\Omega = \mc{M}_c$.
        \item When $V$ is positive definite, $\mc{M}_c$ is bounded for
        sufficiently small $c > 0$. This is not necessarily true when $V$ is not
        positive definite.
        \item If $V$ is radially unbounded (or proper), the set $\mc{M}_c$ is
        bounded for all values of $c$. This is true whether or not $V$ is
        positive definite.
    \end{itemize}
\end{frame}

\begin{frame}
    \frametitle{Corollaries of Krasovskii-LaSalle Theorem}

    \begin{corollary}
        Let $V: D \rightarrow \mathbb{R}$ be a continuously differentiable
        positive definite function on a domain $D$ containing the equilibrium
        point $x=0$, such that $\dot{V}(x) \leq 0$ in $D$. Let $S = \{x \in D:
        \dot{V}(x) = 0\}$ and suppose that no solution can stay identically in
        $S$ other than the trivial solution $x(t) \equiv 0$. Then, the origin is
        asymptotically stable.
    \end{corollary}

    \begin{corollary}
        Let $V: \mathbb{R}^n \rightarrow \mathbb{R}$ be a continuously
        differentiable, radially unbounded, positive defintie function such that
        $\dot{V}(x) \leq 0$ for all $x \in \mathbb{R}^n$. Let $S = \{x \in
        \mathbb{R}^n: \dot{V}(x) = 0\}$ and suppose that no solution can stay
        identically in $S$ other than the trivial solution $x(t) \equiv 0$.
        Then, the origin is globally asymptotically stable.
    \end{corollary}

    Notice that when $\dot{V}(x)$ is negative definite, then $S = \{0\}$.
\end{frame}


\begin{frame}
    \frametitle{Remarks on Krasovskii-LaSalle Theorem}

    \begin{itemize}
        \item The theorem relaxes the negative definiteness requirement of
        Lyapunov's theorem.
        \item It further extends Lyapunov's theorem in three different
        directions.
        \begin{itemize}
            \item It gives an estimate of the RoA, which is not necessarily of
            the form $\mc{M}_c = \{ x \in \mathbb{R}^n: V(x) \leq c\}$. The set
            $\Omega$ of the theorem can be \textsc{any} compact positively
            invariant set.
            \item The theorem can be used in cases where the system has an
            equilibrium set, rather than an isolated equilibrium point.
            \item The function $V$ does not have to be positive definite.
        \end{itemize}
    \end{itemize}
\end{frame}


\begin{frame}
    \frametitle{Example: Stabilization of a Rigid Robot without Gravity}

    \begin{block}{Setup}
        Let $q = (q_1, \ldots, q_n)$ denote the vector of generalized
        coordinates of the robot and $u = (u_1, \ldots, u_n)$ denote the vector
        of generalized forces. The dynamics are given by the Euler-Lagrange
        equations \[ \frac{\dd}{\dd t} \pd{L}{\dot{q}} - \pd{L}{q} = u, \] where
        $L$ is the Lagrangian of the system. Since there is no gravity, the
        potential energy $\mc{P} = 0$ can be taken. Thus, \[ L = K =
        \frac{1}{2}\dot{q}^\top M(q) \dot{q}. \] $M(q) \in \mathbb{S}_{++}^n$ is
        called the \textbf{inertia matrix}. There exist positive constants
        $\alpha$ and $\beta$ such that \[ 0 < \alpha \leq
        \lambda_{\text{min}}\left[M(q)\right] \leq
        \lambda_{\text{max}}\left[M(q)\right] \leq \beta, \;\; \forall q. \]
    \end{block}
\end{frame}

\begin{frame}
    \frametitle{Example: Stabilization of a Rigid Robot without Gravity}

    \begin{block}{The Euler-Lagrange equations}
         With $L=K$, we have \[ \sum_{j=1}^n m_{ij}(q)\ddot{q}_j + \sum_{j=1}^n
         \sum_{k=1}^n c_{ijk}(q) \dot{q}_j \dot{q}_k = u_i, \;\; i = 1, \ldots,
         n, \] where \[ c_{ijk} = \frac{1}{2} \left( \pd{m_{ik}}{q_j} +
         \pd{m_{ij}}{q_k} - \pd{m_{jk}}{q_i} \right) \] are called the
         \textit{Christoffel symbols}. Compactly, we have \[ M(q) \ddot{q} +
         C(q,\dot{q})\dot{q} = u, \] where the $(i,j)^{\text{th}}$ element of
         $C(q,\dot{q})$ is \[ c_{ij}(q,\dot{q}) = \sum_{k=1}^n
         c_{ijk}(q)\dot{q}_k. \]
    \end{block}
\end{frame}


\begin{frame}
    \frametitle{Example: Stabilization of a Rigid Robot without Gravity}

    \begin{block}{State equations and na\"{i}ve control}
        Introduce the state variables $x = q$, $y = \dot{q}$ so that 
        \[ \dot{x} = y, \qquad \dot{y} = \left[ M(x) \right]^{-1} \left[ u -
        C(x,y)y \right]. \]

        Suppose we want to asymptotically stabilize the state $(x,y)$ to a
        desired value $(x_d, 0)$. Let us try the na\"{i}ve control law
        \[ u = -K_p(x - x_d) - K_dy, \] where $K_p, K_d \in \mathbb{S}_{++}^n$.
        The closed-loop dynamics become \[ \dot{x} = y, \qquad \dot{y} = 
        -\left[ M(x) \right]^{-1}\left[ K_p(x-x_d) + K_dy + C(x,y)y \right]. \]
    \end{block}
\end{frame}


\begin{frame}
    \frametitle{Example: Stabilization of a Rigid Robot without Gravity}

    \begin{block}{Lyapunov analysis}
        Consider the Lyapunov function candidate \[ V = \frac{1}{2} \left[y^\top
        M(x) y + (x-x_d)^\top K_p (x-x_d) \right]. \] The first term is the
        kinetic energy, while the second term is the potiential energy due to
        proportional feedback. Note that
        \[ \frac{\dd}{\dd t}\left[ m_{ij}(x) \right] = \sum_{k=1}^n
        \pd{m_{ij}(x)}{x_k}y_k. \] Define the $(i,j)^{\text{th}}$ element
        of $\dot{M}(x,y) \in \mathbb{R}^{n \times n}$ by the RHS above. Now, 
        \begin{align*}
            \dot{V} &= y^\top M(x) \dot{y} + \frac{1}{2}y^\top\dot{M}(x,y)y + \dot{x}^\top K_p (x - x_d) \\
            &= -y^\top \left[ K_p(x-x_d) + K_dy + C(x,y)y \right] + \frac{1}{2}y^\top \dot{M}(x,y)y + y^\top K_p (x-x_d) \\
            &= -y^\top K_d y + \frac{1}{2}y^\top \left[ \dot{M}(x,y) - 2C(x,y) \right]y = -y^\top K_d y + \frac{1}{2} y^\top D(x,y) y.
        \end{align*}
    \end{block}
\end{frame}

\begin{frame}
    \frametitle{Example: Stabilization of a Rigid Robot without Gravity}
    \only<1>{
    \begin{block}{Skew-symmetry of $D(x,y) := \dot{M}(x,y) - 2C(x,y)$}
        We perform the computations in coordinates
        
        \begin{align*}
            d_{ij} = \dot{m}_{ij} - 2c_{ij} &= \left[ \sum_{k=1}^n 
            \pd{m_{ij}}{x_k} - \left( \pd{m_{ik}}{x_j} + 
            \pd{m_{ij}}{x_k} - \pd{m_{jk}}{x_i} \right) \right]y_k
        \end{align*}
    \end{block}
    }
    \onslide<2->{
    \begin{block}{Skew-symmetry of $D(x,y) := \dot{M}(x,y) - 2C(x,y)$}
        \begin{align*}
            d_{ij} = \dot{m}_{ij} - 2c_{ij} &= \left[ \sum_{k=1}^n 
            \cancel{\pd{m_{ij}}{x_k}} - \left( \pd{m_{ik}}{x_j} + 
            \cancel{\pd{m_{ij}}{x_k}} - \pd{m_{jk}}{x_i} \right) \right]y_k \\
            &= \sum_{k=1}^n \left( \pd{m_{jk}}{x_i} - \pd{m_{ik}}{x_j} \right)y_k.
        \end{align*}
        %
        Interchanging $i$ and $j$ gives
        \[ d_{ji} = \sum_{k=1}^n \left( \pd{m_{ik}}{x_j} - \pd{m_{jk}}{x_i} \right)y_k = -d_{ij} \]
    \end{block}
    }
\end{frame}

\begin{frame}
    \frametitle{Example: Stabilization of a Rigid Robot without Gravity}

    \begin{block}{Lyapunov analysis -- resumed}
        Hence $D$ is skew-symmetric and hence $y^\top D y = 0$, so that
        \[\dot{V} = -y^\top K_d y \leq 0. \] The set $E$ of Krasovskii-LaSalle
        theorem is given by \[ E = \{ (x,y) \in \mathbb{R}^n \times
        \mathbb{R}^n: \dot{V} \equiv 0 \} = \mathbb{R}^n \times \{0\}. \]
        Suppose $(x(t), y(t))$ is a trajectory that lies entirely in $E$. Then
        \[ y \equiv 0 \implies \dot{y} \equiv 0 \implies K_p(x-x_d) \equiv 0
        \implies x \equiv x_d, \;\; \forall t \geq 0. \] Hence $E$ contains no 
        trajectories of the system other than the equilibrium $(x_d, 0)$. It 
        follows from the Krasovskii-LaSalla theorem that this equilibrium is 
        \textsc{globally asymptotically stable}.
    \end{block}
\end{frame}

\endgroup