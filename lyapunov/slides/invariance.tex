\section{The Invariance Principle}

\begin{frame}[standout, plain, noframenumbering]
    The Invariance Principle

    % \medskip

    % \footnotesize
    % Sam Greydanus \quad Misko Dzamba \quad Jason Yosinski
\end{frame}

\begingroup
\small


\begin{frame}
    \frametitle{Intuition: Damped Pendulum}

    \begin{columns}
        \begin{column}{0.5\textwidth}
            \begin{align*}
                \dot{x}_1 &= x_2, \\
                \dot{x}_2 &= -a \sin{x_1} - bx_2^2.
            \end{align*}
        \end{column}
        \begin{column}{0.5\textwidth}
            \underline{Lyapunov function candidate}\\[0.75ex]
            $ V(x) = a(1 - \cos{x_1}) + \frac{1}{2}x_2^2. $
        \end{column}
    \end{columns}
    \[ \dot{V}(x) = -bx_2^2 \leq 0. \]

    \begin{itemize}
        \item $\dot{V}(x) < 0$ if and only if $x_2 \neq 0$.
        \item For the system to maintain $\dot{V}(x) = 0$, it has to stay on
        $x_2=0$.
        \item Unless $x_1=0$, this is impossible:
        \[ x_2(t) \equiv 0 \; \Rightarrow \; \dot{x}_2 \equiv 0 \; \Rightarrow
        \; \sin{x_1(t)} \equiv 0. \]
        \item Hence, on the segment $-\pi < x_1 < \pi$ of the $x_2=0$ line, the
        system can maintain $\dot{V}(x) = 0$ only at the origin $x=0$.
        \item Therefore, $V(x(t))$ must decrease towards $0$ and, consequently, 
        \[ x(t) \xrightarrow{t \to \infty} 0. \]
    \end{itemize}
\end{frame}


\endgroup